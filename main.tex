\documentclass{sig-alternate-05-2015}
\usepackage{amsmath}
\usepackage{amssymb}
\DeclareMathOperator*{\argmax}{argmax}
\DeclareMathOperator*{\argmin}{argmin}

\usepackage{mathrsfs}

\begin{document}

\setcopyright{acmcopyright}

\title{Improving Collaborative Filtering with\\
Long-Short Interest Model}

\numberofauthors{1}
\author{
\alignauthor
Chao Lv \quad
Lili Yao \quad
Yansong Feng \quad
Dongyan Zhao\titlenote{Corresponding author.}\\
\affaddr{Institute of Computer Science and Technology}\\
\affaddr{Peking University, Beijing 100871, China}\\
\email{\{lvchao, yaolili, fengyansong, zhaodongyan\}@pku.edu.cn}
}

\maketitle

\begin{abstract}
Collaborative filtering (CF) has been widely employed within
recommender systems in many real-world situations.
The basic assumption of CF is that items liked by the same user would be similar and
users like same items would share similar interest.
But it is not always true since the user's interest changes over time.
It should be more reasonable to assume that
if these items are liked by the same user in the same time period,
there is a strong possibility that they are similar,
but the possibility will shrink if the user likes they in different time period.
In this paper, we propose a long-short interest model (LSIM) based on
the new assumption to improve collaborative filtering.
In special, we introduce a neural network based language model
to extract the sequential features on user's preference over time.
Then, we integrate the sequential features to solve the rating prediction task
in a feature based collaborative filtering framework.
Experimental results on three MovieLens datasets demonstrate that
our approach can achieve the state-of-the-art performance.
\end{abstract}

\category{H.3.3}{Information Search and Retrieval}{Information Filtering}
\category{H.2.8}{Database Management}{Data Mining}
\keywords{Recommender System; Collaborative Filtering; Long-Short Interest Model;}

\section{Introduction}
In the modern era of information overload,
recommender system (RS) has become more and more popular in many real-world situations.
Recommender system aims to help users find the items,
they are more likely to be interested in,
from huge amounts of candidates.
Lots of websites (e.g. Amazon, Netflix, Alibaba and Hulu) use recommender system to
target customers and provide them with useful information.
An excellent recommendation system can effectively increase the amount of sales.
For instance, 80\% of movies watched on Netflix
come from their recommender system \cite{gomez2015netflix}.

A widely used setting of recommender system \cite{ricci2011introduction} is to
predict the rating a user will evaluate on a new item (such as a movie)
given the past rating history of the users.
Lots of classical recommendation methods have been proposed
during the last decade, and they can be categorized into two classes:
content based methods and collaborative filtering based methods.
Content based methods \cite{pazzani2007content} take advantage of
user profiles and item properties for recommendation.
While collaborative filtering based approaches \cite{su2009survey} utilize
the past interactions or preferences, such as users' ratings on items,
without using user or product content information for recommendation.
Collaborative filtering based approaches have attracted more attention
due to their impressive performance, and developed for many years and
keep to be a hot area in both academia and industry.

\begin{figure}[htbp]
    \centering
    \includegraphics[scale=0.7]{images/example.pdf}
    \caption{The preference records of user whose id is $5988$ in MovieLens-1M dataset,
    which are sorted by their rated time.}
    \label{fig:example}
\end{figure}

Collaborative filtering assumes that items liked by the same user would be similar and
users like same items would share similar interest.
However, it is not always true because the user's interest changes over time.
For example, given a user in MovieLens-1M dataset whose id is $5988$,
Figure \label{fig:example} shows the movies he watched sorted by the rating time.
We can find that this user liked watching comedy movies in April 2004 and
changed to love watching drama movies in June 2004.
Thses movies are going to be treated similar in conventional collaborative filtering,
but they are not in actual.
A more reasonable assumption, aka long-short interest assumption, should be that
items liked by the same user in the same time period have a higher possibility
to be similar than items liked by the same user in different time period.

\begin{figure*}[htbp]
    \centering
    \includegraphics[scale=0.38]{images/example2.pdf}
    \caption{Paragraph2vec learns vector representations of sentences and words
    based on the word order
    while LSIM extracts sequential features of users and items
    based on the rating order.
    }
    \label{fig:example2}
\end{figure*}

Inspired by paragraph2vec algorithm \cite{le2014distributed} for
learning vector representations of words which take advantage of
a word order observed in a sentence,
we introduce a long-short interest model (LSIM) to extract sequential features
of users and items based on the new assumption.
As illustrated in Figure \ref{fig:example2}, user is simliar with the sentence,
both of them contains a sequence follow some order,
and items are similar with words because both of them follow the law
that the more close they are, the more similar they are.
To verify the effectiveness of the learned sequential features of users
and items, we integrate them as side information to solve the rating prediction task
in a feature based collaborative filtering framework.

The main contributions of this paper include:
(1) We introduce a long-short interest model (LSIM) to extract sequential features
of users and items based on the long-short interest assumption.
(2) We demonstrate the effectiveness of the sequential features
via integrating them as side information to solve the rating prediction task.
(3) Experiments on three public MovieLens shows LSIM can achieve the state-of-the-art performance.

The rest of the paper is organized as follows. Section 2 gives an overview of the related work.
Then, we describe our long-short interest model and the feature based collaborative filtering
framework in Section 3. The experimental results as well as the comparisons with
baseline system are shown in Section 4.
Finally, we conclude the paper and outline our future work in Section 5.

\section{Related Work}
Our work is closely related to collaborative filtering and
neural network language model.
We will discuss them in the following subsections.

\subsection{Collaborative Filtering}
Collaborative filtering based methods can mainly divided into three categories:
user-based collaborative filtering, item-based collaborative filtering and
model-based collaborative filtering.
User-based collaborative filtering \cite{resnick1994grouplens} recommends
items liked by users who are simliar with you
while item-based collaborative filtering \cite{sarwar2001item} aims to
recommend items similar with ones you liked in the past.
Matrix factorization (MF) is the most popular model-based collaborative filtering methods,
their success at the Netflix competition \cite{koren2009matrix, bennett2007netflix}
have demonstrated their amazing strength,
and lots of variants of it have been proposed in the following works.

Basically, the given ratings matrix $\mathbf{R} \in \mathbb{R}^{N*M}$
consisting of the item preferences of the users can be decomposed as
a product of two low dimensional matrices $\mathbf{U} \in \mathbb{R}^{N*K}$
and $\mathbf{V} \in \mathbb{R}^{K*M}$.
$\mathbf{U}$ could be treated as a user-interest matrix while
$\mathbf{U}$ could be treated as a item-interest matrix.
$K$ is the amount of interest.
The decomposition can be carried out by a variety of methods
such as singular value decomposition (SVD) based approaches \cite{mazumder2010spectral},
non-negative matrix factorization approach \cite{lee2001algorithms}
and regularized alternative least square (ALS) algorithm \cite{zhou2008large}.
Meanwhile, non-linear algorithms are proposed to catch subtle factors,
such as Non Linear Probabilistic Matrix Factorization \cite{lawrence2009non},
Factorization Machines \cite{rendle2010factorization} and
Local Low Rank Matrix Approximation \cite{lee2013local}.
However, these mothodes group users and treat items they rated equally,
which will lose the sequential features to describe the long-short interest.

Matrix factorization methods suffer from the cold start problem,
i.e. what recommendations to make when a new user/item arrives in the system.
Another problem often presented in many real world applications is data sparsity.
Incorporating side information has shown promising performance
in collaborative filtering in such scenarios.
In recent years, deep learning \cite{hinton2006reducing, hinton2006fast}
has attracted a lot of attention due to its amazing performance
to learn representations on various tasks,
especially in computer vision and natural language processing.
Hence, some works make use of deep learning to learn effective
features from side information to improve the performance of recommender system,
such as \cite{salakhutdinov2007restricted, van2013deep, wang2015collaborative, li2015deep}.

Neural Networks have attracted little attention in the collaborative filtering community.
Salakhutdinov \textit{et al.} \cite{salakhutdinov2007restricted} were the first
work to tackle the Netflix challenge using Restricted Boltzmann Machines (RBM).
They modified the RBM as a two-layer undirected graphical model
consisting of binary hidden units and softmax visible units,
and tested their model on the Netflix dataset and
showed a comparable result with the start-of-the-art.
On music recommendation, Van \textit{et al.} \cite{van2013deep}
directly use Convolutional Neural Network (CNN) to learn effective representations
of songs and use them in content-based collaborative filtering framework.
Wang \textit{et al.} \cite{wang2015collaborative} 
directly coupled matrix factorization with deep learning models,
and proposed a hierarchical Bayesian model called Collaborative Deep Learning (CDL)
which tightly couples Stacked Denoising AutoEncoders (SDA) \cite{vincent2008extracting} and
Collaborative Topic Regression (CTR) \cite{wang2011collaborative} to solve the cold start problem.
Li \textit{et al.} \cite{li2015deep} proposed a generate learning framework
to combine rating matrix and side information, they used 
Probabilistic matrix factorization (PMF) \cite{salakhutdinov2011probabilistic} and
marginalized Stacked Denoising AutoEncoders (mSDA) \cite{chen2012marginalized}
in their approach, which is close to CDL but more efficient and scalable.
Meanwhile, their model can learn deep features for both items and users
while CDL only extracts deep features for items.
In our work, although sequential features learned by LSIM are used as side information
in the feature based collaborative framework, but actually they aims to describe
the long-short interest, and cann't be utilized to solve the cold start problem.

\subsection{Nerual Network Language Model}
Traditional language model uses a one-hot representation to represent each word
as a feature vector, where these feature vectors have the same length as the size
of vocabulary, and the position that corresponds to the observed word is equal to 1,
and 0 otherwise. However, this approach often exhibits significant limitations
in practical tasks, suffering from high dimensionality and severe data sparsity.

Mikolov \textit{et al.} \cite{mikolov2013efficient, mikolov2013distributed} proposed
the word2vec algorithm to address these issues. They take advantage of the word order
in text documents, explicitly modeling the assumption that closer words in the word
sequence are statistically more dependent, and have generalized the classic n-gram
language models by using continuous variables to represent words in a vector space.
The continuous bag-of-words (CBOW) and skip-gram (SG) language models are highly
scalable for learning word representations from large-scale corpora.
Le \textit{et al.} \cite{le2014distributed} followed the above work and proposed
the paragraph2vec algorithm to simultaneously learn vector representations of sentence
and words by considering the sentence as a ``global context".
Our long-short interest model shares similar idea with paragraph2vec algorithm,
but we aim to simultaneously learn vector representations of user and items
correspondingly by considering the user as a ``global context".

\section{Our Approach}
In this section,
we first describe the definition of the rating prediction task and
the notation we are going to use in this paper.
Then we introduce our long-short interest model to extract the relationship
between users and items based on the local interest and global interest.
In the last, we utilize the embedding information as pseudo side information
in the machine learning model to do the final prediction.

\subsection{Problem Definition}
Given $N$ users and $M$ items, the rating $r_{ij}$ is the rating given by
the $i^{th}$ user for the $j^{th}$ item.
In the common real-world situations,
users usually rate on a fraction of items, not on the whole items.
Therefore,
those ratings entail a sparse matrix $\mathbf{R} \in \mathbb{R}^{N \times M}$.
The goal of recommender system is to make a prediction on the missing ratings.
Based on that, we will know the preference of a user on the items he never rates,
and recommend high score items to him.

Matrix Factorization is a classic method to solve this problem.
It aims to find a $K$ dimensional low rank matrix $\mathbf{\hat{R}} \in \mathbb{R}^{N \times M}$
where $\mathbf{\hat{R}} = \mathbf{U} \mathbf{V}^\mathrm{T}$ with
$\mathbf{U} \in \mathbb{R}^{N \times K}$ and $\mathbf{V} \in \mathbb{R}^{M \times K}$
are two matrices of rank $K$ encoding a dense representation of the users and items with

\begin{equation}
\begin{aligned}
	\argmin_{\mathbf{U},\mathbf{V}}
	\sum_{(i,j) \in \mathcal{K}(\mathbf{R})}
	( r_{ij} - \overline{\mathbf{u}}_i^{\mathrm{T}} \overline{\mathbf{v}}_j ) ^ 2 +
	\lambda ( \left\| \overline{\mathbf{u}}_i \right\|_{Fro}^2 +
	\left\| \overline{\mathbf{v}}_j \right\|_{Fro}^2 )
\end{aligned}
\end{equation}

where $\mathcal{K}(\mathbf{R})$ is the set of indices of known ratings,
$\overline{\mathbf{u}}_i$ and $\overline{\mathbf{v}}_j$
are the corresponding line vectors of $\mathbf{U}$ and $\mathbf{V}$,
$\lambda$ is the coefficient that controls the influence of L2 regularization,
and $\left\| \cdot \right\|_{Fro}$ is the Frobenius norm.

Matrix Factorization techniques, as the main thrust behind collaborative filtering,
have been developed for many years and keep to be a hot area in both academia and industry.
But it suffers from the problem of cold-start,
i.e. what recommendations to make when a new user or a new item arrives in the system
with the fact that their have no ratings information.
Meanwhile, when the ratings information is too sparse,
the performance of matrix factorization will decrease rapidly.

Side information, such as the users' profiles and the items' properties,
is a effective resource to remit this problem.
In the next section, we will propose a embedding model
to extract latent side information based on the local interest and global interest.

\subsection{Embedding Model}
Collaborative Filtering aims at estimating the ratings
a user would have given to all other items he never interact with
by using the ratings of all the other users.
The basic assumption of collaborative filtering is that
items liked by the same user would be similar
or users like same items would share similar interest.
However, in real-world situations,
it is not always hold because users' interest may change over
a long time period.
Meanwhile, the interest distribution of a user in a fixed time period
are stable and don't change too much.

To describe the phenomenon that interest changes over time,
we propose the definition of \textbf{local interest} and \textbf{global interest}.

\begin{itemize}
\item \textbf{local interest} reflects the interest distribution of a user
in a short and fixed time period.
\item \textbf{global interest} reflects the interest distribution of a user
in a long time period.
\end{itemize}

And we believe that item in the same local interest (i.e. in the same time period)
of a user have a higher possibility to be similar than
items in different local interest.

\begin{figure}[htbp]
	\centering
	\includegraphics[scale=0.5]{images/embedding.pdf}
	\caption{A Example for Local Interest and Global Interest}
	\label{fig:embedding}
\end{figure}

For example, as illustrated in the left of Figure \ref{fig:embedding},
given a certain $\mathbf{user}$, we can see he interact with five items
($\mathbf{item1}$ to $\mathbf{item5}$) in the last.
They are list on the time line according to the time order.
Let's assume that the time window size \footnote{We use time window to judge time proximity
instead of extract timestamp here for convenience.} is set to $2$,
Therefore $\mathbf{item1}$ and $\mathbf{item2}$ are in the same local interest,
$\mathbf{item2}$ and $\mathbf{item3}$ are in the same local interest, and so on.
Meanwhile $\mathbf{item2}$ and $\mathbf{item5}$ are in different local interest.
That means $\mathbf{item2}$ has a higher possibility to be similar with $\mathbf{item1}$
than $\mathbf{item5}$.

To describe the similarity preference in mathematical sense,
we project users and items into a low-dimensionality space,
and change the interest similarity into spacial distance.
$\mathbf{item2}$ is close to $\mathbf{item1}$ and far away from $\mathbf{item5}$
in the low-dimensionality space, as shown in the right of Figure \ref{fig:example}.

\begin{figure}[htbp]
	\centering
	\includegraphics[scale=0.55]{images/doc2vec.pdf}
	\caption{Embedding Model for Extracting Interest Similarity from Users and Items}
	\label{fig:doc2vec}
\end{figure}

Inspired by paragraph2vec algorithm \cite{le2014distributed} for learning
vector representations of words which take advantage of
a word order observed in a sentence,
we choose to use neural network based language model
to embed this interest similarity information.
The embedding model simultaneously learns vector representations of users and items
by considering the user as a global context,
and the architecture of the embedding model is illustrated in Figure \ref{fig:doc2vec}.

The training data set was derived from users interaction timeline $T$,
which comprises users $u_i (i=1,2,...,N)$ and their interacted items ordered by the interacted time,
$v_{i_1}$, $v_{i_2}$, ..., $v_{i_{L_i}}$,
where $L_i$ denotes number of items interacted by user $u_i$,
which is much less than the amount of items $M$.

More formally, objective of the embedding model is to
maximize the log-likelihood over the set of $T$ of all the interaction timeline,

\begin{equation}
\begin{aligned}
	\sum_{i=1}^{N} \bigg( &p(u_i | v_{i_1}, v_{i_2}, ..., v_{i_{L_n}}) + \\
	                      &\sum_{j=1}^{L_i} p(v_{i_j} | v_{i_{j-c}}, ..., v_{i_{j-1}}, v_{i_{j+1}},..., v_{i_{j+c}}, u_i) \bigg)
\end{aligned}
\end{equation}

where $c$ is the time window size.
$p(u_i | v_{i_1}, v_{i_2}, ..., v_{i_{L_i}})$ is the probability to generate
the global interest of $u_i$ based on all items he interacted.
The prediction task is typically done via a multiclass classifier,
such as softmax. There, we have

\begin{equation}
	p(u_i | v_{i_1}, v_{i_2}, ..., v_{i_{L_i}}) =
	\frac
	{
		exp ( \overline{\mathbf{v}}_{1}^{\mathrm{T}} \mathbf{v}_{u_i}^{'} )
	}
	{
		\sum_{k=1}^{V} exp ( \overline{\mathbf{v}}_{1}^{\mathrm{T}} \mathbf{v}_{k}^{'} )
	}
\end{equation}

where $\mathbf{v}_{u_i}^{'}$ is the output vector representation of $u_i$,
$V$ is the set of the whole items,
and $\overline{\mathbf{v}}_{1}$ is averaged input vector representation of all the items
interacted by user $u_i$, i.e.

\begin{equation}
	\overline{\mathbf{v}}_{1} = \frac{\sum_{j=1}^{T_i} \mathbf{v}_{v_{i_j}}}{T_i}
\end{equation}

$p(v_{i_j} | v_{i_{j-c}}, ..., v_{i_{j-1}}, v_{i_{j+1}},..., v_{i_{j+c}}, u_i)$
is the probability to generate the interest distribution of $v_{i_j}$
based on items in the same local interest and the user's global interest.
Similarly, using softmax multiclass classifier we have

\begin{equation}
	p(v_{i_j} | v_{i_{j-c}}, ..., v_{i_{j-1}}, v_{i_{j+1}},..., v_{i_{j+c}}, u_i) =
	\frac
	{
		exp( \overline{\mathbf{v}}_{2}^{\mathrm{T}} \mathbf{v}_{v_{i_j}}^{'} )
	}
	{
		\sum_{k=1}^{V} exp( \overline{\mathbf{v}}_{2}^{\mathrm{T}} \mathbf{v}_{k}^{'} )
	}
\end{equation}

where $\mathbf{v}_{v_{i_j}}^{'}$ is the output vector representation of $v_{i_j}$,
$V$ is the set of the whole items,
and $\overline{\mathbf{v}}_{2}$ is averaged input vector representation of items
int the same local interest and corresponding global interest $u_i$.

\begin{equation}
	\overline{\mathbf{v}}_{2} = \frac{ \mathbf{v}_{v_{i_{j-c}}} + ... + \mathbf{v}_{v_{i_{j-1}}} + 
	\mathbf{v}_{v_{i_{j+1}}} + ... + \mathbf{v}_{v_{i_{j+c}}} + \mathbf{v}_{u_i} }{2c+1}
\end{equation}


\subsection{Learning Model}
With the help of embedding model, we can obtain
the vector representation of users and items
based on the local interest and global interest.
We use the following notations to describe these representations:

\begin{itemize}
	\item $\tilde{\mathbf{u}}_i$ reflects the interest distribution of the $u_i$
	\item $\tilde{\mathbf{v}}_j$ reflects the interest distribution of the $v_j$
	\item $D$ is the length of those representation
\end{itemize}

There are two kinds of side information in collaborative filtering:
users' profiles and items' properties.
Various kinds of information can be used to model these factors.
In some sense, the representation of users and items from the embedding model
could be treat as a kind of side information,
because it distinguish the similarity and difference between users and items,
just like other side information does.
But it doesn't reflect a actual meaning, so we call it the \textbf{latent side information}. 

Inspired by svdfeature algorithm \cite{chen2012svdfeature}, we utilize the feature-based
collaborative filtering to take advantage of the latent side information.
First, we generate the user feature $\alpha_{i}$ and the item feature $\beta_{j}$
for user $u_i$ and item $v_j$.

\begin{equation}
\alpha_{i} = \{ OneHot(u_i), \tilde{\mathbf{u}}_i \}
\end{equation}
\begin{equation}
\beta_{j} = \{OneHot(v_j), \tilde{\mathbf{v}}_j \}
\end{equation}

where $OneHot(u_i)$ is the one hot representation of $u_i$
and $OneHot(v_j)$ is the one hot representation of $v_j$.

Then we predict the rating for $u_i$ and $v_j$

\begin{equation}
\hat{r} = \sum_{i=1}^{n} \alpha_{i} b_{i}^{(u)} + \sum_{i=1}^{m} \alpha_{i} b_{i}^{(i)} +
\left( \sum_{i=1}^{n} \alpha_{i} \textbf{p}_{i} \right) ^ \mathrm{T}
\left( \sum_{i=1}^{m} \beta_{i} \textbf{q}_{i} \right)
\end{equation}

where $\textbf{p}_{i} \in \mathbb{R}^d$ and $\textbf{q}_{i} \in \mathbb{R}^d$
are $d$ dimensional latent factors associated with each feature.
$b_{i}^{(u)}$ and $b_{i}^{(i)}$ are bias parameters that directly influence the prediction.


\section{Experiment}

\subsection{Baseline Models}
We use two popular techniques that are widely used
in both academia and industry as our baseline models.

\textbf{LibMF} is an open source tool
for approximating an incomplete matrix using the product of
two matrices in a latent space.
It provides solvers for real-valued matrix factorization,
binary matrix factorization, and one-class matrix factorization.
It also supports parallel computation in a multi-core machine
using CPU instructions (e.g., SSE) to accelerate vector operations.
Its paper \cite{chin2015fast} won the best paper award
in RecSys 2013.

Factorization machines (FM) are a generic approach that
allows to mimic most factorization models by feature engineering.
This way, factorization machines combine the generality of
feature engineering with the superiority of factorization models
in estimating interactions between categorical variables of large domain.
\textbf{LibFM} \cite{rendle2012factorization} is a software implementation
for factorization machines that features
Stochastic Gradient Descent (SGD) and
Alternating Least Squares (ALS) optimization as well as
Bayesian inference using Markov Chain Monte Carlo (MCMC).
We select ALS optimization in our baseline experiments.

\subsection{Dataset}
We conduct experiments on three benchmark datasets MovieLens-1M, MovieLens-10M and MovieLens-20M
\footnote{http://grouplens.org/datasets/movielens/},
which are commonly used for evaluating collaborative filtering algorithms.
The MovieLens-1M dataset consists of about 1 million ratings of 6040 users and 3706 movies,
and each rating is an integer between 1 (worst) and 5 (best).
The MovieLens-10M dataset consisits of about 10 million ratings of 69878 users and 10677 movies,
the MovieLens-20M dataset consisits of about 20 million ratings of 138493 users and 26744 movies,
and each rating ranges from 0.5 (worst) to 5.0 (best) step by 0.5.
The ratings are highly sparse.
Table \ref{tab:statistics} summarizes the statistics of three datasets.

\begin{table}[htpb]
	\centering
	\caption{Statistics of three datasets}
	\label{tab:statistics}
	\begin{tabular}{|l|c|c|c|c|}
		\hline
		\textbf{Dataset} & \textbf{\#Users} & \textbf{\#Items} & \textbf{\#Ratings} & \textbf{Sparsity} \\
		\hline
		ML-1M  & 6,040    & 3,706  & 1,000,209   & 95.53\% \\
		ML-10M & 69,878   & 10,677 & 10,000,054  & 98.66\% \\
		ML-20M & 138,493  & 26,744 & 20,000,263  & 99.46\% \\
		NetFlix & 480,189 & 17,770 & 100,480,507 & 98.82\% \\
		\hline
	\end{tabular}
\end{table}

\subsection{Metrics}
We employ the root mean squared error (RMSE) and mean absolute error (MAE) as the evaluation metric.
RMSE and MAE are defined as:
\begin{equation}
	RSME = \sqrt{ \frac{1}{N} \sum_{i,j} I_{ij} (R_{ij} - \hat{R}_{ij})^2 }
\end{equation}
\begin{equation}
	MAE = \frac{1}{N} \sum_{i,j} I_{ij} |R_{ij} - \hat{R}_{ij}|
\end{equation}
where $N$ is the total number of ratings in the test set,
$R_{ij}$ is the ground-truth rating of user $i$ for item $j$,
$\hat{R_{ij}}$ denotes the corresponding predicted rating,
and $I_{ij}$ is abinary matrix that indicates the ratings in the test set.




\begin{table*}[htpb]
	\centering
	\caption{MSRE}
	\label{tab:msre}
	\begin{tabular}{|l|c|c|c|c|}
		\hline
		\textbf{Algorithms} & \textbf{MovieLens-1M} & \textbf{MovieLens-10M} & \textbf{MovieLens-20M}  & \textbf{NetFlix} \\
		\hline
		LibMF      & 0.8554 $\pm$ 0.0013 & 0.8090 $\pm$ 0.0004 & 0.8023 $\pm$ 0.0004 & 0.8630 $\pm$ 0.0002 \\
		LibFM-SGD  & 0.8641 $\pm$ 0.0015 & 0.8022 $\pm$ 0.0013 & 0.7945 $\pm$ 0.0023 & \\
		LibFM-MCMC & 0.8460 $\pm$ 0.0011 & 0.7866 $\pm$ 0.0004 & 0.7787 $\pm$ 0.0005 & \\
		LibFM-ALS  & 0.8453 $\pm$ 0.0015 & 0.7936 $\pm$ 0.0004 & 0.7860 $\pm$ 0.0004 & 0.8406 $\pm$ 0.0001 \\
		LSIM       &  &  &  & \\
		\hline
	\end{tabular}
\end{table*}

\begin{table*}[htpb]
	\centering
	\caption{MAE}
	\label{tab:msre}
	\begin{tabular}{|l|c|c|c|c|}
		\hline
		\textbf{Algorithms} & \textbf{MovieLens-1M} & \textbf{MovieLens-10M} & \textbf{MovieLens-20M}  & \textbf{NetFlix} \\
		\hline
		LibMF      & 0.6816 $\pm$ 0.0008 & 0.6311 $\pm$ 0.0003 & 0.6229 $\pm$ 0.0004 & 0.6822 $\pm$ 0.0002 \\
		LibFM-SGD  & 0.6674 $\pm$ 0.0027 & 0.6127 $\pm$ 0.0034 & 0.6016 $\pm$ 0.0036 & \\
		LibFM-MCMC & 0.6661 $\pm$ 0.0008 & 0.6039 $\pm$ 0.0002 & 0.5941 $\pm$ 0.0005 & \\
		LibFM-ALS  & 0.6609 $\pm$ 0.0010 & 0.6068 $\pm$ 0.0002 & 0.5971 $\pm$ 0.0003 & 0.6481 $\pm$ 0.0002 \\
		LSIM       &  &  &  & \\
		\hline
	\end{tabular}
\end{table*}


\section{Conclusions}


\section{Acknowledgments}
The work reported in this paper is supported by the National Natural Science Foundation of China Grant 61370116.
We thank anonymous reviewers for their beneficial comments.
We also thank Feifan Fan and Yue Fei for valuable suggestions related to this paper.

\bibliographystyle{abbrv}
\bibliography{sigproc}

\end{document}
